%% Adaptado a partir de :
%%    abtex2-modelo-trabalho-academico.tex, v-1.9.2 laurocesar
%% para ser um modelo para os trabalhos no IFSP-SPO

\documentclass[
    % -- opções da classe memoir --
    12pt,               % tamanho da fonte
    openright,          % capítulos começam em pág ímpar (insere página vazia caso preciso)
    %twoside,            % para impressão em verso e anverso. Oposto a oneside
    oneside,
    a4paper,            % tamanho do papel. 
    % -- opções da classe abntex2 --
    %chapter=TITLE,     % títulos de capítulos convertidos em letras maiúsculas
    %section=TITLE,     % títulos de seções convertidos em letras maiúsculas
    %subsection=TITLE,  % títulos de subseções convertidos em letras maiúsculas
    %subsubsection=TITLE,% títulos de subsubseções convertidos em letras maiúsculas
    % para pacote url reconhecer hifens como separador
    hyphens,
%    paginasA3,  % indica que vai utilizar paginas em A3 
    % Opções que não devem ser utilizadas na versão final do documento
    draft,              % para compilar mais rápido, remover na versão final
    MODELO,             % indica que é um documento modelo então precisa dos geradores de texto
    TODO,               % indica que deve apresentar lista de pendencias 
    % -- opções do pacote babel --
    english,            % idioma adicional para hifenização
    brazil           % o último idioma é o principal do documento
    ]{ifsp-spo-inf-cpti} % ajustar de acordo com o modelo desejado para o curso

% ---
% Pacotes básicos 
% ---
%\usepackage[utf8]{inputenc}     % Codificacao do documento (conversão automática dos acentos)
% ---

%\usepackage{style}
        


% --- 
% CONFIGURAÇÕES DE PACOTES ADICIONAIS UTEIS
% --- 
\usepackage{pdfpages}			% para incluir arquivos pdf no documento


% ---
% Informações de dados para CAPA e FOLHA DE ROSTO
% ---
\titulo{TÍTULO DO TRABALHO}
\autor{AUTOR DO TRABALHO}

\preambulo{Modelo canônico de trabalho monográfico acadêmico em conformidade com
as normas ABNT apresentado à comunidade de usuários \LaTeX.}

\data{DATA DO TRABALHO}

% Definir o que for necessário e comentar o que não for necessário
% Utilizar o Nome Completo
\orientador{ORIENTADOR}
\coorientador{COORIENTADOR}

% ---


% informações do PDF
\makeatletter
\hypersetup{
        %pagebackref=true,
        pdftitle={\@title}, 
        pdfauthor={\@author},
        pdfsubject={\imprimirpreambulo},
        pdfcreator={LaTeX with abnTeX2 using IFSP model},
        pdfkeywords={abnt}{latex}{abntex}{abntex2}{IFSP}{\ifspprefixo}{trabalho acadêmico}, 
        colorlinks=true,            % false: boxed links; true: colored links
        linkcolor=blue,             % color of internal links
        citecolor=blue,             % color of links to bibliography
        filecolor=magenta,              % color of file links
        urlcolor=blue,
        bookmarksdepth=4
}
\makeatother
% --- 


% ----
% Início do documento
% ----
\begin{document}

% Retira espaço extra obsoleto entre as frases.
\frenchspacing 

%somente para o exemplo, fica primeiro
%\input{00-teste}
\input{00-info}

% -- lista de pendencias gerada pelo todonotes
% -- altere opções do usepackage para remover na versão final....
\listoftodos
\todo[inline]{remover lista de todo da versão final...}
\newpage

% ----------------------------------------------------------
% ELEMENTOS PRÉ-TEXTUAIS
% ----------------------------------------------------------
\pretextual

% ---
% Capa
% ---
\imprimircapa

\newcounter{todocounter}
\newcommand{\todonum}[2][]
{\stepcounter{todocounter}\todo[#1]{\thetodocounter: #2}}


\todonum[inline]{ajustar titulo do trabalho}
\todonum[inline]{ajustar autor}
\todonum[inline]{ajustar data}
\todonum[inline]{ajustar preambulo}
\todonum[inline]{ajustar curso}
\todonum[inline]{ajustar disciplina}
\todonum[inline]{ajustar departamento}
\todonum[inline]{ajustar orientador/coorientador/professor(es)}
% ---

% ---
% Folha de rosto
% (o * indica que haverá a ficha bibliográfica)
% ---
\imprimirfolhaderosto
%\imprimirfolhaderosto*
% ---

% Quando registrado na biblioteca
%\input{pre-fichacatalografica}

%Caso necessário
%\input{pre-errata}

%Obrigatório para trabalhos com bancas oficiais
%\input{pre-aprovacao}

% ---- opcionais 
\input{pre-dedicatoria}
\input{pre-agradecimentos}
\input{pre-epigrafe}

% -- resumo obrigatório
% ---
% RESUMOS
% ---

% resumo em português
\setlength{\absparsep}{18pt} % ajusta o espaçamento dos parágrafos do resumo
\begin{resumo}

 Essa proposta de trabalho é referente a primeira etapa de um projeto da matéria de Recursos Humanos em Tecnologia da Informação. Nesse estudo foram analisados processos e mecanismos de contratação da empresa Ernst & Young Global Limited, dividido em três partes, as quais se debruçam sobre, a empresa de um modo geral, o recrutamento e uma crítica sobre o processo seletivo, respectivamente. O intuito desse projeto é de identificar, analisar e debater, ferramentas e técnicas utilizadas na contratação, de forma a fomentar um senso crítico fundamentado na teoria, a qual pudemos ter em sala de aula.


 \textbf{Palavras-chaves}: recursos humanos. projeto. estudo.
\end{resumo}

% resumo em inglês
\begin{resumo}[Abstract]
 \begin{otherlanguage*}{english}
This work proposal is related to the first stage of a Human Resources in Information Technology project. In this study, the hiring processes and mechanisms of the company Ernst & Young Global Limited were analyzed, divided into three stages, which focus on, the company, the recruitment process and a criticism about it, respectively. The purpose of this project is to identify, analyze and debate, tools and techniques used in recruitment process, in order to stimulate a critical sense based on theory, which we were able to have in the classroom.


   \vspace{\onelineskip}
 
   \noindent 
   \textbf{Key-words}: human resources. project. study.
 \end{otherlanguage*}
\end{resumo}


% ---
% inserir lista de ilustrações
% ---
\pdfbookmark[0]{\listfigurename}{lof}
\listoffigures*
\cleardoublepage
% ---

% ---
% inserir lista de tabelas
% ---
\pdfbookmark[0]{\listtablename}{lot}
\listoftables*
\cleardoublepage
% ---

% ---
% inserir lista de quadros
% ---
\pdfbookmark[0]{\listofquadrosname}{loq}
\listofquadros*
\cleardoublepage
% ---

\input{pre-siglas}

\input{pre-simbolos}
\todo[inline]{Remover lista de simbolos se não for necessário}


% ---
% inserir o sumario
% ---
\pdfbookmark[0]{\contentsname}{toc}
\tableofcontents*
\cleardoublepage
% ---


% ----------------------------------------------------------
% ELEMENTOS TEXTUAIS
% ----------------------------------------------------------
\textual

\input{textos-introducao}
\input{textos-revisao-literatura}

% Para facilitar a manutenção é sempre melhore criar um arquivo por capitulo, para exemplo isso não é necessário 
\input{textos-desenvolvimento}
% exemplos de escrita LaTeX
\input{exemplos/exemplos}



\input{textos-conclusao}

% ----------------------------------------------------------
% Finaliza a parte no bookmark do PDF
% para que se inicie o bookmark na raiz
% e adiciona espaço de parte no Sumário
% ----------------------------------------------------------
\phantompart

% ----------------------------------------------------------
% ELEMENTOS PÓS-TEXTUAIS
% ----------------------------------------------------------
\postextual
% ----------------------------------------------------------

% ----------------------------------------------------------
% Referências bibliográficas
% ----------------------------------------------------------
\bibliography{referencias,exemplos/abntex2-doc-abnt-6023}

\input{pos-glossario.tex}

\input{pos-apendices}
\input{pos-anexos}


%---------------------------------------------------------------------
% INDICE REMISSIVO - Quando necessário 
% As palavras indexadas devem ser definidas com \index{} no texto
%---------------------------------------------------------------------
\phantompart
\printindex
\todonum[inline]{remover indice remissivo se não for necessário}

%---------------------------------------------------------------------

\end{document}