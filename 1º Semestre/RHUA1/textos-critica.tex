\chapter[Crítica]{Análise crítica dos mecanismos de contratação}

\section{Primeira Etapa}

A etapa de inscrição e avaliação curricular online é muito importante para o processo seletivo da EY, pois através da divulgação feita por meio de plataformas de seleção de candidatos, as vagas são direcionadas e tornam-se mais acessíveis aos possíveis candidatos, economizando tempo dos colaboradores e recursos para realizar a seleção de perfis que se adéquam aos requisitos desejados. Também é viável para o candidato interessado na vaga, pois toda a plataforma fornece o suporte necessário durante essa etapa, dando sugestões de vagas de acordo com o currículo da pessoa. A divulgação do programa de \textit{Trainees} apenas por meio das redes sociais faz com que um público muito grande tenha maior conhecimento da empresa, mas nem todos os interessados são compatíveis com a vaga, tornando um processo de filtragem muito mais lento e assim tomando tempo dos próprios colaboradores que realizam o contato com os indivíduos.
A avaliação curricular e inscrição online dos candidatos mostra-se bem completa, apesar de que nas redes sociais atrai um grande número de pessoas, muitas vezes não-compatíveis. Durante essa etapa há uma filtragem das vagas através da plataforma, fazendo com que as próximas etapas sejam mais objetivas e não consumam tanto tempo e recursos da empresa. 


\section{Segunda Etapa}
A parte da apresentação institucional da empresa EY é fundamental para o processo seletivo, pois é de suma importância que o candidato conheça melhor a empresa e esteja alinhado com os interesses da mesma como a missão, visão e valores. Neste momento o indivíduo perceberá se suas necessidades serão supridas com o trabalho a ser realizado dentro organização, visto que as necessidades do mesmo nem sempre serão atendidas somente pelo dinheiro, mas também por outros fatores. O conhecimento sobre a organização será refletido na motivação do futuro funcionário, já que o mesmo terá conhecimento do trabalho esperado e como ele vai realizar de uma forma que os resultados sejam favoráveis para si, tanto quanto, para a organização posteriormente durante a construção de sua carreira.
Quanto a redação presencial, a mesma mostra-se muito útil na filtragem de dados do candidato, suas áreas de interesse, conhecimento de escrita e de entendimento a fazer algo que foi apresentado, visto que o tema é delimitado e tem que ser seguido, porém, a redação presencial demanda muito tempo e recursos para ser realizada, visto que nesta etapa do processo seletivo há muitos candidatos, cerca de cinquenta pessoas, em média. Sendo assim seria mais viável para a organização antecipar a etapa de testes online a fim de eliminar os candidatos que não cumprem com o ideal desejado pela vaga, dessa maneira reduzindo o número de candidatos durante essa etapa e assim economizando bastante tempo de todos os colaboradores responsáveis pelo processo.

\section{Terceira Etapa}
Essa etapa é uma das etapas mais simples e eficazes de todo o processo de forma geral, pois analisa todos os candidatos que não preenchem os requisitos mínimos desejados para fazer parte do programa através dos testes de conhecimentos aplicados. Apesar de sua simplicidade e sua eficácia, essa etapa poderia ser antecipada no projeto e até mesmo fazer parte da inscrição do candiado, visto que é uma etapa online de classificação e eliminação de candidatos. Caso fosse executada antes da fase de apresentação da empresa da empresa, que é presencial, economizaria muitos recursos investidos pela empresa para a estrutura da apresentação, visto que alguns candidatos que passariam por essa fase já seriam excluídos do processo. 

\section{Quarta Etapa}
Nessa etapa do processo seletivo da EY não há muitas considerações a serem feitas, pois pelas descrições e experiência de um dos integrantes do grupo o processo de dinâmica do grupo é bem construído e executado.  Como toda dinâmica de grupo a apresentação é parte importante, e, nesse caso em específico, ela serve como complemento ao teste de personalidade e gera um melhor entendimento dos resultados. Essa etapa do processo é muito importante e a forma como ela é conduzida demonstra a preocupação da gestão e reduzir o número de candidatos para essa fase é imprescindível, fugindo do erro comumente visto em processos que fazem suas seletivas com muitos candidatos e assim torna a avaliação mais lenta. Outro ponto a ser destacado nessa fase é em como os grupos são divididos, fazendo uma divisão que não leva em consideração a área almejada pelo candidato, retirando a pessoa de sua zona de conforto e fazendo-a lidar com pessoas de outras áreas de conhecimento, o que também gera trabalhos com maior qualidade, pois terão pontos de vista de áreas diferentes, o que torna a experiência do candidato nessa etapa muito mais rica do que ela normalmente é e permite uma sutil apresentação de como se dá o dia a dia na empresa, dessa forma, preparando para situações que ele venha enfrentar caso passe no processo seletivo. Também é possível constatar que abertura de perguntas para todos os participantes da dinâmica também serve como um mecanismo de avaliação dos candidatos, pois nessa fase você pode observar comportamentos que dentro de um ambiente empresarial podem ser nocivos, como por exemplo, alguém fazer uma pergunta com o claro objetivo de constranger ou de prejudicar o grupo que está apresentando ou até mesmo identificar se as pessoas estavam de fato prestando atenção do conteúdo apresentado. De maneira geral essa etapa pode ser considerada muito positiva dentro do processo seletivo, possuindo o momento onde é dado o \textit{feedback} para todos os participantes, ressaltando os pontos positivos e negativos do grupo e de cada integrante ao fim da dinâmica, que é recebido de uma forma que evita colocar os candidatos em situações desconfortáveis, não expondo eles a nenhuma situação vexatória e evita que ele se sinta desencorajado a participar novamente do processo seletivo futuramente


\section{Quinta Etapa}

A quinta e última etapa é considerada mais importante do processo, visto que é a partir dela que os candidatos são classificados, também sendo analisada como um ponto positivo de processo seletivo da empresa.
A entrevista com o gerente técnico, tem como objetivo analisar os requisitos técnicos do candidato. O ponto positivo é, pelo gerente ser da mesma área pretendida pelo candidato, consegue ser avaliado o nível do conhecimento do candidato em determinados assuntos de fora precisa. Por ser um processo seletivo é de \textit{Trainees}, o candidato não é submetido a nenhuma análise técnica de seu conhecimento, visto que o cargo é justamente para ser treinado e desenvolvido. Dentro dessa entrevista também pode ser avaliada, de forma mais intensa durante todo o processo, a atitude do candidato, que apesar de ser observada e todos os momentos do processo, tende a ser mais perceptível durante a conversa.
A entrevista com o sócio se baseia em analisar requisitos importantes para o segmento de atuação da empresa e qual será o segmento adequado para aquele candidato, sendo também um dos momentos onde a atitude do candidato pode ser melhor analisada.
Ao final das duas entrevistas, o gerente, o sócio e o entrevistador da dinâmica de grupo se reúnem a fim de debater cada candidato, visando classificá-los. Essa análise é um ponto de elogio no processo, visto que são levadas em consideração três opiniões diferentes referente a mesma pessoa e suas competências, para que assim a avaliação seja o mais assertiva possível. Nesse momento já é atribuído o projeto inicial de atuação de casa candidato com base em todas suas competências analisadas até o momento. Outro ponto a se ressaltar é que essa etapa faz com que o candidato se sinta mais próximo de pessoas importantes na empresa, além de fazer com que o processo seja justo quanto a sua classificação.




% -----------------------------------------------

\chapter[Conclusão]{Conclusão}

% ---
Tendo em vista as etapas observadas, ficou evidente que o processo seletivo atual é longo e possuí um com custo elevado, mas que preza em selecionar pessoas alinhadas com a cultura da empresa.
O processo seletivo de \textit{Trainees} da EY é amplo e consistente, visa encontrar os melhores candidatos de acordo com o perfil da empresa, e também, permite ao candidato um conhecimento da mesma, criando um ambiente ideal para ambas as partes, o que pode ser observado nas posições em que a empresa figura como uma das melhores empresas para se trabalhar, devido ao forte critério de contratação de pessoas alinhadas aos seus valores.
Apesar do sucesso do programa de \textit{Trainees} em encontrar candidatos qualificados, o tempo investido poderia ser diminuído, sem alterar os ideais desejados durante todo o processo, economizando recursos e mantendo o padrão de qualidade de todo o processo.

\section{Processo Seletivo de \textit{Trainees} Reformulado}

\subsection{Primeira Etapa: Inscrição, Avaliação Curricular e Testes Online}

Etapa inteiramente direcionada à divulgação, todo o processo de inscrição e avaliação de conhecimento específico  de caráter eliminatório do candidato, visando excluir do processo todos os candidatos que não preenchem os requisitos mínimos da vaga.

\subsection{Segunda Etapa: Apresentação Institucional e Redação}

Etapa direcionada à introdução da empresa aos candidatos, visando permitir o candidato uma visão mais clara dos valores e ideais praticados pela empresa e seleção da área desejada pelo candidato, seguindo de maneira inalterada nesses quesitos, e redação, a fim de verificar se o candidato esta alinhado aos ideais da empresa.

\subsection{Terceira Etapa: Dinâmica de Grupo}

Etapa direcionada à avaliação de atitude dos candidatos, visando novamente selecionar os candidatos mais alinhados aos valores praticados pela empresa, seguindo o modelo já praticado no processo seletivo original.  

\subsection{Quarta Etapa: Entrevistas Finais}

Etapa direcionada à avaliação de atitude e conhecimentos específicos dos candidatos, visando  selecionar e direcionar os candidatos com base em seus conhecimentos já adquiridos, novamente seguindo o modelo já praticado no processo seletivo original.  

\subsection{Quinta Etapa: Processos Admissionais}

Etapa direcionada à avaliação do candidato feita pela empresa, ressaltando todos os pontos observados e entregando ao candidato um \textit{feedback} de todo o processo realizado pelo mesmo, dando sequência aos processos de contratação dos candidatos selecionados